\documentclass[12pt]{article}
\usepackage{amsmath}
\title{Modern Problems in Nuclear Physics: neutron-star EoS and high-mass twins}
\date{}

\begin{document}
\maketitle


\section{Solution of the TOV equation}

Write a program for solving the TOV equation

\begin{gather}
	\frac{dP(r)}{dr} = - \frac{ [E(r) + P(r)] [M(r) + 4 \pi r^3 P(r)]} {r (r - 2 M(r))}, \\
	\frac{dM(r)}{dr} = 4 \pi r^2 E(r)
\end{gather}

for a given set of central densities or pressures. Here the units are $G = c = 1$; you may find the conversion factors and detailed discussion on how to solve the equations in \cite{Glend}. Use the EoS in the file "MKVOR.dat" to obtain the solution. The following plots are expected as a result:
\begin{itemize}
	\item Neutron star gravitational mass $M$ in units of $M_\odot = 1.4766$ km as a function of the central density $n_c$ in units of $n_0 = 0.16$ fm$^{-3}$. The NS with a maximum mass should be clearly seen from the plot.
	\item Neutron star mass $M(n_c)$ vs. NS radius $R(n_c)$ for the same set of the central densities.
\end{itemize}


\section{Third branch of compact stars}

The piecewise-polytropic EoS is defined as:

\begin{gather}
P(n) = 
\left\{
\begin{tabular}{cc}
$\kappa_1 n^{\Gamma_1}$, & $n < n_{12}$\\
$P_c$,& $n_{12} < n < n_{23}$ \\
$\kappa_3 n^{\Gamma_3}$, & $n_{23} < n$.
\end{tabular}
\right.
\end{gather}

\begin{gather}
E(n) = 
\left\{
\begin{tabular}{cc}
$\dfrac{\kappa_1 n^{\Gamma_1}}{(\Gamma_1 - 1)} + m_1 n$, & $n < n_{12}$\\
$P_c$,& $n_{12} < n < n_{23}$ \\
$\dfrac{\kappa_3 n^{\Gamma_3}}{(\Gamma_3 - 1)} + m_3 n$, & $n_{23} < n$.
\end{tabular}
\right.
\end{gather}

We will employ this assumption about the EoS shape for studying the possible consequences of the first-order phase transition from hadronic to quark matter on the mass-radius diagram. The task in composed of the following steps:

\begin{enumerate}
	\item Implement and test these formulas in the code. 
	
	\item Perform a fit of the hadronic EoS given for the Task 1 using this formula with $n_{12} \to \infty$
	
	{\bf Output:}
	
	\begin{itemize}
		\item Obtain the parameters $\kappa_1, \, \Gamma_1$.
		\item Plot the fitting curve compared with the original.
	\end{itemize}

	
	\item Partially reproduce the results of \cite{Alvarez-Castillo:2017qki} for the values $\Gamma_1 = 4.92$ and $\kappa_1 = 17906.60 \, {\rm MeV} \cdot {\rm fm}^{3 (\Gamma_1 - 1) }$, and $n_{12} = 0.32 \, {\rm fm}^{-3}, \, n_{23} = 0.53 \, {\rm fm}^{-3}$. The mass parameter $m_1 = 938 \, {\rm MeV}$ is the nucleon mass. 
	
	{\bf Output}
	\begin{itemize}
		\item $M(n_c)$, MR diagram
		\item Pressure and density profile inside a star with the maximum mass as a function of the radial coordinate.
	\end{itemize}
	
	\item Write a code for finding $n_{23}$ for given $\kappa_1, \Gamma_1, \kappa_3, \Gamma_3, n_{12}$. For a fixed $n_{12}$ and a fixed "hadronic" EoS vary $\kappa_3, \, \Gamma_3$ to change $n_{23}$.  in order to check the validity of the Seidov's criterion \cite{seidov} for the existence of the third branch:
	
	\begin{eqnarray}
		\frac{\Delta E}{E_{12}} = \frac{1}{2} + \frac{3}{2} \frac{P_c}{E_{12}},
	\end{eqnarray}
	
	where $E_{12} = E(n_{12}), \, E_{23} = E(n_{23})$, and  $\Delta E = E_{23} - E_{12}$.
	
	
	{\bf Output:}
	\begin{itemize}
		\item MR plot for several values of $n_{23}$ like Fig. 2 in \cite{Lindblom:1998dp}.
		\item Comparison of $\Delta E / E_{12}$ with the formula for the critical value of $n_{12}$.
	\end{itemize}
\end{enumerate}

\begin{thebibliography}{widestlabel}
	\bibitem{Glend}
	Glendenning, N.K. 2000, Compact Stars, Nuclear Physics, Particle Physics, and General Relativity, 2nd ed.
	
	%\cite{Alvarez-Castillo:2017qki}
	\bibitem{Alvarez-Castillo:2017qki}
	D.~E.~Alvarez-Castillo and D.~B.~Blaschke,
	%``High-mass twin stars with a multi-polytrope EoS,''
	arXiv:1703.02681 [nucl-th].
	%%CITATION = ARXIV:1703.02681;%%
	
	\bibitem{seidov}
	Z. F. Seidov, 
	Soviet Astronomy, Vol. 15, p.347
	
	\bibitem{Lindblom:1998dp} 
	L.~Lindblom,
	%``Phase transitions and the mass radius curves of relativistic stars,''
	Phys.\ Rev.\ D {\bf 58}, 024008 (1998)
	doi:10.1103/PhysRevD.58.024008
	[gr-qc/9802072].
	%%CITATION = doi:10.1103/PhysRevD.58.024008;%%
	%28 citations counted in INSPIRE as of 18 Mar 2018
	
	
\end{thebibliography}
\end{document}